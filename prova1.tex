\documentclass[exam]{IFCExams}

\title{Fundamentos de Lógica e Algoritmos}
\subject{Estruturas sequenciais e condicionais}
\author{Prof. Me. Hewerton Enes de Oliveira}
\order{1}
%\email{hewerton.oliveira@concordia.ifc.edu.br}
%\place{Instituto Federal Catarinense - Campus Concórdia}

  \begin{document}
  
  \begin{exeenv}{3}
    Analise o algoritmo a seguir e determine qual será a saída apresentada na tela do usuário.
  
    \begin{algorithm}[H]
    \Inicio{
      
      \Declare valor1, valor2, resultado \Int\; 
      
      valor1 = 2\;
      valor2 = 3\;
      resultado = valor2 \% valor1\;
      valor1 = valor2 / valor1\;
      
      resultado = resultado + valor1\;
      
      \eSe{(resultado \% 2) == 0}
      {
      resultado = (valor1 * 2) + 5\;
      }
      {
      resultado = (valor1 - 1) + 5\;
      }
      
      resultado = valor1 + valor2\;
      
      \BlankLine
      
      \Escreva ``O valor do resultado é:'', resultado\;    
    }
    \end{algorithm}
    ~\newline
  \end{exeenv}
  
  
  \question{3}{
  Faça um algoritmo que receba dois valores inteiros (num1 e num2) e mostre para o usuário se num1 é divisível por num2.
  }
  
  ~\newline
  
  \begin{exeenv}{4}
  Uma empresa decidiu dar uma gratificação de Natal a seus funcinários, baseada no número de horas extras e no número de horas que o funcionário
  faltou ao trabalaho. O valor de prêmio é obtido pela consulta à tabela que se segue, na qual: \\ H = número de horas extras - (2/3 * (número de horas falta))
  
  
    \begin{center}
      \begin{tabular}[!h]{| c | c|}
	\hline
	H (Minutos) & Prêmio (R\$) \\ 
	\hline
	>= 2.400 & 500,00\\ 
	\hline
	1.800 <= H > 2.400  & 500,00\\ 
	\hline
	1.200 <= H > 1.800 & 500,00\\ 
	\hline
	600 <= H > 1.200 & 500,00\\ 
	\hline
	>= 2.400 & 500,00\\ 
	\hline
    
      \end{tabular} 
    
    \end{center}
    ~\newline
  \end{exeenv}
 
  \begin{exeenv}{3}
    Faça um algoritmo que receba dois valores inteiros (num1 e num2) e mostre para o usuário se num1 é divisível por num2.
  \end{exeenv}


  
\end{document}
